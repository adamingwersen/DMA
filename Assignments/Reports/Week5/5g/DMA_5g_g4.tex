\documentclass[a4paper,10pt]{article}
\usepackage[utf8]{inputenc}
\usepackage{amsmath}
\usepackage{amsfonts}
\usepackage{amssymb}
\usepackage{algorithm}
\usepackage[noend]{algpseudocode}
\usepackage{program}
\usepackage{amsmath}
\usepackage{graphicx}
\usepackage[T1]{fontenc}
\usepackage{eso-pic}
\usepackage{gensymb}
\usepackage{listings}
\usepackage{float}

\newcommand{\BackgroundPic}{\put(-4,0){\parbox[b][\paperheight]{\paperwidth}{\centering\includegraphics[width=\paperwidth,height=\paperheight]{nat-farve.pdf}}}}

\algnewcommand\True{\textbf{true}\space}
\algnewcommand\False{\textbf{false}\space}
\algdef{SE}[SUBALG]{Indent}{EndIndent}{}{\algorithmicend\ }%
\algtext*{Indent}
\algtext*{EndIndent}

\begin{document} 
	\AddToShipoutPicture*{\BackgroundPic}
	
	\begin{titlepage}
		\thispagestyle{empty}
		\vspace*{5cm}
		\begin{center}
			\Huge \textbf{Diskret Matematik og Algoritmer} \\
			\LARGE \textbf{Aflevering 5g} \\
		\end{center}
		\vspace*{3.5cm}
		\begin{flushleft}
			
		\begin{table}[h!]
			\begin{tabular}{lll}
				Adam Ingwersen,& \\ Peter Friborg,& \\ Aske Fjellerup\\
			\end{tabular}
		\end{table}
			
			
			\vspace{3mm}
			\vspace{3mm}
			Datalogisk  Institut\\
			Københavns Universitet\\
			\vspace{3mm}
			\today\\
			%\vspace*{0.5cm}
			
		\end{flushleft}
	\end{titlepage}

	\title{5g}
	\author{ncog}
	
	\newpage

\newpage

\subsection*{Del 1}
\subsubsection*{(1)}
Her udregnes GCD(Greatest Common Divisor) for tallene (8,5) hhv. (13,8), efter samme notation som anført på s.23 i KBR.

\begin{itemize}
\item{\textbf{GCD(8,5):}}
\item{$8 = 1 \cdot 5 + 3$}
\item{$5 = 1 \cdot 3 + 2$}
\item{$3 = 1 \cdot 2 + 1$}
\item{$2 = 2 \cdot 1 + 0$}
\item{$\rightarrow GCD(8,5) = 1$}
\end{itemize}

\begin{itemize}
\item{\textbf{GCD(13,8):}}
\item{$13 = 1 \cdot 8 + 5$}
\item{$8 = 1 \cdot 5 + 3$}
\item{$5 = 1 \cdot 3 + 2$}
\item{$3 = 1 \cdot 2 + 1$}
\item{$2 = 2 \cdot 1 + 0$}
\item{$\rightarrow GCD(13,8) = 1$}
\end{itemize}

Tallene i figur 1 angiver antallet af nødvendige operationer for at bestemme GCD for en bestemt kombination af tal. GCD(8,5) tager 4 operationer - og derved indsættes tallet 4 i den celle, hvor 5. række og 8. kolonne krydser. For GCD(13,8) skal der stå 5. Se Bilag.


\subsubsection*{(2)}
Lad $T = [t_{1}, t_{2},...,t_{15}]$ være et array, som udfyldes med et worst-case antal operationer for $t_{n}$:
$$
T = [1, 1, 2, 2, 3, 2, 3, 4, 3, 3, 4, 4, 5, 4, 4]
$$

Her observeres det, at hvert "uptick" i køretid forekommer ved Fibonacci-tal(Hvor $[f_{1}=1, f_{2}=1, f_{3}=2, f_{4}=3, f_{5}=5, f_{6}=8, f_{7}=13, f_{8}=21,...$]). Det gælder for Fibonacci-tal, at $f_{k} = f_{k-1} + f_{k-2}$. 

\subsubsection*{(3)}

Euklids algoritme vil opnå worst-case køretid, når de to tal a og b's største divisor er 1- eftersom dette implicerer, det højeste antal mulige delinger af et sæt af tal. 
Dette er altid tilfældet ved Fibonacci tal, der ved første operation i $GCD(f_{k}, f_{k-1})$ vil returnere.
\begin{equation}
f_{k} = 1 \cdot f_{k-1} + f_{k-2} = 1 \cdot f_{k} + f_{k} \mod f_{k-1}
\end{equation}
Givet Fibonacci-tallenes konstruktion, vil denne sekvens altid resultere i, at GCD af to Fibonacci-tal er lig 1. 
Således, for et vilkårligt sæt (a,b) af positive naturlige tal; 
\begin{equation}
\text{for  $n \in \mathbb{N}^{+}$  hvor $ a \geq b \in n$,}
\end{equation}

...til to Fibonacci-tal altid resultere i den højeste køretid for GCB(a,b), når $a=f_{k}, b=f_{k-1}$.

Et generelt eksempel for køretiden af Euklids GCD ved et sæt af to konsekutive Fibonacci-tal præsenteres nedenfor:

\begin{itemize}
\item{\textbf{GCD($f_{k},f_{k-1}$):}}
\item{$f_{k} = f_{k-1} + f_{k} \mod f_{k-1}$}
\item{$f_{k-1} = f_{k-2} + f_{k-1} \mod f_{k-2}$}
\item{$f_{k-2} = f_{k-3} + f_{k-2} \mod f_{k-3}$}
\item{...}
\item{$f_{k-(k-2)} = 2 \cdot f_{1} + f_{k-(k-2)} \mod f_{1} = 2 \cdot 1 + f_{1} \mod f_{k-(k-2)} = 2 + 0$}
\item{$\rightarrow GCD(f_{k},f_{k-1}) = 1$}
\end{itemize}
Givet ovenstående sekvens er det klart, at algoritmen må terminere efter $k-1$ operationer, da både $f_{1} = f_{2} = 1$.  Dette vil i store-O-notation implicere en worst-case når (2) er overholdt:
$$
\text{$t_{n}$  er $ O(n)$}
$$

\subsubsection*{(4)}

Det fremgår af opgavebeskrivelsen, at $n \geq a \geq b > 0$. Dette sætter de nedre grænser for mulige værdier af a og b - men ikke de øvre. En vilkårlig sekvens af Fibonacci-tal der overholder $n > a > b > 1$, vil overholde ulighederne defineret, men vil resultere i en køretid, der vokser i takt med at inputs vokser langs Fibonacci-sekvensen. En konstant køretid opnås udelukkende i best-case; $n \geq a = b > 0$. Herved har vi, at:

$$
\text{$t_{n}$  er $ \Omega(1)$}
$$

\subsubsection*{(5)}

Grafen ser ikke ud til, at indikere en lineær vækst i antal operationer over værdier for $n$. Som funktion af $n$, ser antallet af operationer ud til at være aftagende. Dette skyldes, at ovenstående retfærdiggørelse i store-O notation er 'løs' idet $t_{n} er O(n)$ angiver en øvre grænse - men dette kan præciseres. \newline 
Det blev fastslået, at GCD($f_{k},f_{k-1}$) terminerer efter k-1 operationer. Afstanden mellem Fibonacci-tal stiger eksponentielt over k, vi kan derfor skrive, at $k \approx \gamma log(n)$, hvor gamma er en konstant. Dette kan vi netop fordi, at der for Fibonacci-tal gælder, at mængden n skal stige eksponentielt over k - og for hver ekstra operation kræves det næste tal i sekvensen, som er eksponentielt voksende over n. \newline
Da vi nu ved, at $k$ kan approksimeres ved $\gamma log(n)$, og at køretiden for GCD kan skrives som $O(n)$, må det da også gælde, at $t_{n}$ er $O(log(n))$. Hermed er upper-bound indsnævret. Det er her klart, at med den nedre grænse fundet i (4), må $t_{n}$ ligeledes være $\Omega(log(n))$, da det i selve best-case scenariet er ækvivalent med $ \Omega(1)$. \newline
Idet vi har identificeret en funktion, der tilfredsstiller:
$$
t_{n} = O(log(n)) og t_{n} = \Omega(log(n))
$$
...siger vi, at:
$$
t_{n} = \Theta(log(n))
$$






\end{document}





