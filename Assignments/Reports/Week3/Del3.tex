\documentclass[11pt,a4paper,final]{article}
\usepackage[utf8]{inputenc}
\usepackage{amsmath}
\usepackage{amsfonts}
\usepackage{amssymb}
\usepackage{float}
\usepackage{lipsum}
\usepackage{graphicx}
\usepackage{hyperref}

\begin{document}

\subsection*{Del 3}

Vi betragter følgende udtryk: 
$$
\sum_{k=0}^{n} (2k+1)
$$

Vi ønsker, at lade $k$ starte i nul, med det formål at være i stand til at anvende reglerne fra formelsamlingen. Det ses, at $2\cdot0+1 = 1$, hvorved den første iteration af sumfølgen altid vil være 1 - dette kan bruges, så:
$$
= 1 + \sum_{k=1}^{n} (2k+1)
$$
Herefter kan vi anvende, at $\sum_{k=1}^{n} 1 = n$, samt at $\sum_{k=1}^{n}c \cdot a_{k} = c \cdot \sum_{k=1}^{n}a_{k}$, sådan at:
$$
1 + \sum_{k=1}^{n} (2k+1) = 1 + n + 2 \cdot \sum_{k=1}^{n}k
$$
Hertil ved vi, at $\sum_{k=1}^{n}k = \sum_{k=1}^{n}\frac{n^{2}+n}{2}$. Herved har vi opnået den eksplicitte form for sumfølgen:
$$
1 + n + 2 \cdot \sum_{k=1}^{n}k = 1 + n + 2 \cdot \frac{n^{2}+n}{2} = 
$$
$$
\underline{\underline{1 + 2n + n^{2}}}
$$

\end{document}

