\documentclass[a4paper,10pt]{article}
\usepackage[utf8]{inputenc}
\usepackage{amsmath}
\usepackage{amsfonts}
\usepackage{amssymb}
\usepackage{algorithm}
\usepackage[noend]{algpseudocode}
\usepackage{program}
\usepackage{amsmath}
\usepackage{graphicx}
\usepackage[T1]{fontenc}
\usepackage{eso-pic}
\usepackage{gensymb}
\usepackage{listings}
\usepackage{float}

\newcommand{\BackgroundPic}{\put(-4,0){\parbox[b][\paperheight]{\paperwidth}{\centering\includegraphics[width=\paperwidth,height=\paperheight]{nat-farve.pdf}}}}

\algnewcommand\True{\textbf{true}\space}
\algnewcommand\False{\textbf{false}\space}
\algdef{SE}[SUBALG]{Indent}{EndIndent}{}{\algorithmicend\ }%
\algtext*{Indent}
\algtext*{EndIndent}


\begin{document} 
	\AddToShipoutPicture*{\BackgroundPic}
	
	\begin{titlepage}
		\thispagestyle{empty}
		\vspace*{5cm}
		\begin{center}
			\Huge \textbf{Diskret Matematik og Algoritmer} \\
			\LARGE \textbf{Aflevering 2i} \\
		\end{center}
		\vspace*{3.5cm}
		\begin{flushleft}
			
		\begin{table}[h!]
			\begin{tabular}{lll}
				Adam Frederik Ingwersen Linnemann,& \\ GQR701\\ Hold 4\\
			\end{tabular}
		\end{table}
			
			
			\vspace{3mm}
			\vspace{3mm}
			Datalogisk  Institut\\
			Københavns Universitet\\
			\vspace{3mm}
			\today\\
			%\vspace*{0.5cm}
			
		\end{flushleft}
	\end{titlepage}

	\title{2i}
	\author{Adam}
	
	\newpage

\newpage


\subsection*{Del 1}
Antallet af inversioner kan bestemmes ved at tælle det samlede antal 'ikke-sorterede' placeringer af elementerne i A. For arrray bestående af følgende $n = 6$ elementer; [2,1,8,4,3,6] er der sammenlagt \underline{5} inversioner. 
\\
\subsection*{Del 2}
En generel formel for at finde det maksimale antal inversioner, k, i et array bestående af n vilkårlige elementer kan bestemmes ved:
$$
k = \frac{n(n-1)}{2}
$$

Dette kan bl.a. skrives på følgende måde ved iterativ pseudokode:

\begin{algorithm}
\caption{Maksimale antal inversioner givet n}
\begin{algorithmic}[1]
\Function {MaxInv}{$n$}
\State \parbox[t]{.7\linewidth}{v = 0}
\State \parbox[t]{.7\linewidth}{for i = 1 to n}
\Indent
\State \parbox[t]{.7\linewidth}{v = (i-1) + v}
\EndIndent
\State \Return {v}
\EndFunction
\end{algorithmic}
\end{algorithm}

\subsection*{Del 3}

Denne delopgave kan løses på adskillige måder - både rekursivt og iterativt. Her er valgt en iterativ tilgang, der bygger på lineær søgning. Algoritmen opererer på følgende måde:

\begin{enumerate}
\item{Positionér starten af den første løkke (i=0) ved A's 0'te element}
\item{Positionér herefter starten af den anden løkke foran i (j=i+1)}
\item{Lad j iterere fra j=1 til n over nedenstående betingelse:}
\begin{enumerate}
\item{Hvis værdien af det i'te element er strengt større end det j'te, da:}
\item{Denotér variablen 'inv' med sin hidtige værdi plus 1}
\end{enumerate}
\item{Afslut ved at returnere den akkumulerede værdi for 'inv'}
\end{enumerate}

Algoritmen er beskrevet i et mere koncist format i pseudokoden nedenfor. 

\begin{algorithm}[H]
\caption{Tæl antal inversioner i A}
\begin{algorithmic}[2]
\Function {CountInv}{$A, n$}
\State \parbox[t]{.7\linewidth}{inv = 0}
\State \parbox[t]{.7\linewidth}{for i = 0 to n - 2}
\Indent
\State \parbox[t]{.7\linewidth}{for j = i+1 to n - 1}
\Indent
\State \parbox[t]{.7\linewidth}{if $A[i] > A[j]$}
\State \parbox[t]{.7\linewidth}{inv = inv + 1}
\EndIndent
\EndIndent
\State \Return {inv}
\EndFunction
\end{algorithmic}
\end{algorithm}


En alternativ løsning til den 'fladpandede' lineære søgning ville være, at anvende merge-sort med et if-statement og en tælle-variabel i den afsluttende 'merge'-del.

\subsection*{Del 4}

For at analysere algoritmens køretid, anvendes tilgangen beskrevet i CLRS 2.2. Her er vi umiddelbart kun interesserede i de led, der påvirker køretiden, når $n\leftarrow \infty$, så at sige. Af denne årsag, betragtes udelukkende de iterative led i $CountInv(A,n)$. Leddene har en multiplikativ relation. Det sidste for-loop aftager i  antallet af gentagelser som funktion af i's placering, mens det første for-loop gentages $n-1$ gange.
\newline

$$
rt \approx \Theta((n-1)\cdot\frac{(n-1)((n-1)+1)}{2}) =
$$
$$
\Theta((n-1)\cdot\frac{n^{2}- n}{2}) =
$$
$$
\Theta(\frac{n\cdot(n-1)^{2}}{2})
$$
\\
\begin{figure}[H]
\centering
\includegraphics[scale = 0.5]{plotR1.png}
\caption{Algoritmens køretid sammenlignet med $n^{2}$}
\end{figure}

Ved at plotte køretiden mod f.eks. $n^{2}$, ses det, at algoritmen ved store n vil opnå eksponentielt højere køretider i takt med at $n\leftarrow\infty$. Dette må anses for at være en direkte effekt af det voksende antal mulige sammensætninger. 
For at konstruere plottet er der anvendt R-kode, som er vedlagt i bilag. 

\subsection*{Bilag}
\lstinputlisting[language=R]{plotRuntime.R}
 
\end{document}
